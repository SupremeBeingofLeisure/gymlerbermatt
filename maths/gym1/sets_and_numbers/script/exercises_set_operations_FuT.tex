\documentclass[11pt,a4paper]{article}
\usepackage[T1]{fontenc}
\usepackage[german]{babel}
%\usepackage{lmodern}
\usepackage{amsmath, amsthm,amsfonts,amssymb, mathtools}


\usepackage{enumerate, enumitem}
\usepackage{graphicx}
\usepackage{tikz, tkz-base,tkz-euclide}


\usepackage{venndiagram}
\usepackage{multicol}


\usepackage{fancyhdr}
\usepackage{fancybox}
\setlength\parindent{0pt}

\usepackage[most]{tcolorbox} %for answerbox


\newcommand{\F}{\mathbb{F}}
\newcommand{\C}{\mathbb{C}}
\newcommand{\N}{\mathbb{N}}
\newcommand{\Q}{\mathbb{Q}}
\newcommand{\A}{\mathbb{A}}
\newcommand{\B}{\mathbb{B}}
\newcommand{\G}{\mathbb{G}}
\newcommand{\R}{\mathbb{R}}
\newcommand{\Z}{\mathbb{Z}}
\newcommand{\K}{\mathbb{K}}


\theoremstyle{definition}
\newtheorem{aufg}{Aufgabe}
\newtheorem{aufg*}[aufg]{\hspace*{-0.40cm}* Aufgabe}
\newtheorem{aufgE}[aufg]{\hspace*{-0.65cm}\Ovalbox{E} Aufgabe}



\usepackage{hyperref}

\newcommand{\assignmentTitle}{Megenoperationen}
\newcommand{\assignmentQuestionName}{Aufgabe} % The word to be used as a prefix to question numbers; example alternatives: Problem, Exercise


\title{}
\date{\today}
\author{Tina Furer}


\hoffset=-0.5in
\voffset=-1in

\textwidth=16cm
\topmargin=1.5cm
\headheight=1cm \headsep=1cm \textheight=24cm

\pagestyle{fancy}
\markright{\textbf{\huge\assignmentTitle}  \hfill 05.09.2023  } %\\ \vspace{10pt} Tina Furer
\fancyfoot{} 



\newcommand{\answerbox}[1]{
	\begin{tcolorbox}[colback=white, breakable, enhanced]
		\vphantom{L}\vspace{\numexpr #1-1\relax\baselineskip} % \vphantom{L} to provide a typesetting strut with a height for the line, \numexpr to subtract user input by 1 to make it 0-based as this command is
	\end{tcolorbox}
}



\begin{document}
\begin{aufg}
	Seien $\A=\{1,4,6,2\}, \B=\{1,3,5\}$ und $\C=\{4,5,6,7,8\}$. Gib die folgenden Mengen an:
	\begin{enumerate}[label=(\alph*)]
		\begin{multicols}{4}
			\item $\A\cup\B$
			\item $(\A\cup\B)\cap\C$
			\item $(\C\setminus\A)\cup\B$
			\item $\C\setminus(\B\cap\A)$
		\end{multicols}
	\end{enumerate}
\end{aufg}
\begin{aufg}
	Sei $\G=\N$ die Grundmenge und 
	\begin{align*}
		\A &= \{x\in\G\mid x \text{ ist gerade}\}\\
		\B &= \{x\in\G\mid x \text{ ist kleiner als } 50\}\\
		\C &= \{1,3,5,7,9\}
	\end{align*}
	Teilmengen von $\G$. Ermittle
	\begin{enumerate}[label=(\alph*)]
		\begin{multicols}{4}
			\item $\A\cup\B$
			\item $\overline{\A}$
			\item $\overline{\C}\setminus\A$
			\item $\overline{\B}\cap\A$
		\end{multicols}
	\end{enumerate}
	
\end{aufg}

	
\begin{aufg} Welche Mengen sind in den folgenden Venn Diagrammen schattiert?
	\begin{enumerate}[label=(\alph*)]
		\begin{multicols}{2}
		\item 
		\begin{venndiagram2sets}[tikzoptions={scale=0.75,thick}, labelA=$\mathbb{A}$,labelB=$\mathbb{B}$]\fillANotB\fillBNotA \end{venndiagram2sets}
		\item 
		\begin{venndiagram3sets}[tikzoptions={scale=0.75,thick}, labelA=$\mathbb{A}$,labelB=$\mathbb{B}$,labelC=$\mathbb{C}$]\fillACapB\fillACapC\fillBCapC\end{venndiagram3sets}
		\end{multicols}
		\begin{multicols}{2}
		\item 
		\begin{venndiagram3sets}[tikzoptions={scale=0.75,thick}, labelA=$\mathbb{A}$,labelB=$\mathbb{B}$,labelC=$\mathbb{C}$]\fillACapBNotC\end{venndiagram3sets}
		\item 
		\begin{venndiagram3sets}[tikzoptions={scale=0.75,thick}, labelA=$\mathbb{A}$,labelB=$\mathbb{B}$,labelC=$\mathbb{C}$]\fillACapBNotC\fillC\end{venndiagram3sets}
	\end{multicols}
	\end{enumerate}

\end{aufg}

	
\begin{aufg}
	Welche der folgenden Aussagen sind richtig?
	\begin{enumerate}[label=(\alph*)]
	\begin{multicols}{2}
		\item $\emptyset\subset\mathcal{P}(\emptyset)$
		\item $\emptyset\in\mathcal{P}(\emptyset)$
		\item $\N\cap\Q=\N_0$
		\item $(\Q\setminus\N_0)\cup\Z = \Z$
		\item $\N_0\setminus\N=\{0\}$
		\item $\mathcal{P}(\{1,2\})=\{\{1\},\{2\},\{1,2\}\}$
		\item $\mathcal{P}(\{1,2\})\cap\N=\emptyset$
		\item $\mathcal{P}(\mathcal{P}(\emptyset))=\{\emptyset,\{\emptyset\}\}$
	\end{multicols}
	\end{enumerate}
\end{aufg}


\begin{aufg}
	Von den Schüler*innen einer Klasse spielen 6 kein Instrument. 10 Schüler*innen spielen Violine und 7 spielen Klavier. Ferner gibt es 12 Flötenspieler*innen in der Klasse, von denen alle mit Ausnahme von dreien noch mindestens ein weiteres Instrument spielen, nämlich 6 Violine und 5 Klavier. Von den Violinist*innen spielen 3 kein weiteres Instrument.\\
	Wie viele Schüler*innen...
	\begin{enumerate}[label=(\alph*)]
		\item zählt die Klasse?
		\item spielen nur Klavier?
		\item spielen alle drei Instrumente?
		\item spielen Violine und Klavier?
	\end{enumerate}
\end{aufg}

\begin{aufg}
	Sei $\A$ eine Menge mit $n$ Elementen. Wie viele Elemente hat die Potenzmenge $\mathcal{P}(\A)$?
\end{aufg}



\begin{aufg}
	Gelten die folgenden Gleichungen?
	\begin{enumerate}[label=(\alph*)]
		\item $(\A\cup\B)\cup\C=\A\cup(\B\cup\C)$
		\item $(\A\cap\B)\cup\C=\A\cap(\B\cup\C)$
		\item $\G\setminus(\A\cup\B)=(\G\setminus\A)\cup(\G\setminus\B)$
			\item $\G\setminus(\A\cup\B)=(\G\setminus\A)\cap(\G\setminus\B)$
	\end{enumerate}
\end{aufg}
	

\begin{aufg}
	Was ist die Produktmenge folgender Mengen
	\begin{enumerate}[label=(\alph*)]
		\item $\A=\{0,1\}$ und $\B=\{6,7\}$
		\item $\A=\{blau, rot, gelb\}$ und $\B=\{Haus, Boot\}$.
		\item $\emptyset$ und $\A=\{1,2\}$
		\item $\Z$ und $\N$
	\end{enumerate}
\end{aufg}

\begin{aufg}
		Sei $\A$ eine Menge mit $n$ Elementen und $\B$ eine Menge mit $m$ Elementen. Wie viele Elemente hat $\A\times\B$?
\end{aufg}
	
	
	
	
	
	
	
	
	
	
	
	
	
	
\end{document}