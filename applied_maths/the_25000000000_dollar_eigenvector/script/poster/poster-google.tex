\documentclass[final,%hyperref={pdfpagelabels=false}
]
{beamer}
\usepackage{grffile}
\mode<presentation>{\usetheme{gymkl}}
\usepackage[german]{babel}
\usepackage[applemac]{inputenc}
\usepackage[OT1]{fontenc}
\usepackage{amsmath,amsthm, amssymb, latexsym}
\usepackage{listings}
%\usepackage{times}\usefonttheme{professionalfonts}  % obsolete
\usefonttheme[onlymath]{serif}
\boldmath
\usepackage[orientation=portrait,size=a3,scale=1.4,debug]{beamerposter}

\usepackage{tikz-er2}
\usepackage{framed}
\usetikzlibrary{arrows}
\usetikzlibrary{positioning}
\usetikzlibrary{shadows}
% change list indention level
% \setdefaultleftmargin{3em}{}{}{}{}{}


%\usepackage{snapshot} % will write a .dep file with all dependencies, allows for easy bundling

\usepackage{array,booktabs,tabularx}
\newcolumntype{Z}{>{\centering\arraybackslash}X} % centered tabularx columns
\newcommand{\pphantom}{\textcolor{ta3aluminium}} % phantom introduces a vertical space in p formatted table columns??!!

% Command, um Tabellen-Spalten anzupassen
\newcommand{\spaltenheight}{\rule{0mm}{2.5ex}}
\newcommand{\spaltenwidth}{\rule{3cm}{0mm}}
\newcommand{\spaltensep}{\\[1ex]}

\listfiles

%%%%%%%%%%%%%%%%%%%%%%%%%%%%%%%%%%%%%%%%%%%%%%%%%%%%%%%%%%%%%%%%%%%%%%%%%%%%%%%%%%%%%%
\graphicspath{{../pictures/}}
 
\title{{\LARGE Der milliardenschwere Eigenvektor}\\
{\Large Lineare Algebra bei Google}}
\author{Jorma Wassmer}
\institute[AM]{Schwerpunktfach Angewandte Mathematik}
\date[20.09.13]{20. September 2013}

%%%%%%%%%%%%%%%%%%%%%%%%%%%%%%%%%%%%%%%%%%%%%%%%%%%%%%%%%%%%%%%%%%%%%%%%%%%%%%%%%%%%%%
\newlength{\columnheight}
\setlength{\columnheight}{105cm}


%%%%%%%%%%%%%%%%%%%%%%%%%%%%%%%%%%%%%%%%%%%%%%%%%%%%%%%%%%%%%%%%%%%%%%%%%%%%%%%%%%%%%%
\begin{document}
\lstset{language=Mathematica}
\begin{frame}
  \begin{columns}
    % ---------------------------------------------------------%
    % Set up a column 
    \begin{column}{.49\textwidth}
      \begin{beamercolorbox}[center,wd=\textwidth]{postercolumn}
        \begin{minipage}[T]{.95\textwidth}  % tweaks the width, makes a new \textwidth
          \parbox[t][\columnheight]{\textwidth}{ % must be some better way to set the the height, width and textwidth simultaneously
            % Since all columns are the same length, it is all nice and tidy.  You have to get the height empirically
            % ---------------------------------------------------------%
            % fill each column with content 
\vskip1.5ex
            \begin{block}{Man googelt}
                        \begin{columns}
            \begin{column}{0.5\textwidth}
		Ein Teil des Erfolgs von Google ist ihr \emph{PageRank-Algorithmus}, der die Relevanz jeder Webseite quantifiziert und daher die \glqq guten\grqq\ Links zuerst anzeigt.
			  \end{column}
	  \begin{column}{0.3\textwidth}     
	\begin{center}
	  \includegraphics[width=1\textwidth]{google}
	\end{center}
		  \end{column}
	  \end{columns}
            \end{block}
\vskip1.5ex
            \begin{block}{Auftraggeber}
		Jeder Webpage-Designer ist interessiert, wie PageRank die Relevanz berechnet. Dadurch kann er die Chance beeinflussen, dass seine Seite in Google m�glichst weit vorne aufgelistet und damit von vielen Leuten besucht wird.
            \end{block}
            
\vskip1.5ex
            \begin{block}{Ranking System entwickeln}
		 Das Ziel ist, eine der Grundideen des Webpage-Ranking von Google kennenzulernen. Vorerst stellen wir uns selbst allgemein die Frage:
		
		\begin{quote}
			Wie definiert man in einem Web eine sinnvolle Quantifizierung der Relevanz von Webpages?
		\end{quote}		
            \end{block}
\vskip1.5ex
            \begin{block}{Ans�tze}
            Unsere Ans�tze sind teilweise vergleichbar mit einem demokratischen System: Webpages \glqq voten\grqq\ mit Links andere Webpages.
%            \begin{columns}
%            \begin{column}{0.5\textwidth}
\begin{description}              
            \item[Anzahl Backlinks:] Ein Link bedeutet ein Votum f�r eine Webpage. Sei $x_k$ gleich der Anzahl Backlinks auf Page $k$. In der Abbildung ist $x_1=2$, $x_2=1$, $x_3=3$ und $x_4=2$. Dieser simple Ansatz ignoriert, dass ein Link zu einer Page von einer relevanten Page mehr z�hlt, als von einer weniger relevanten.
            
            \item[Gewichtete Backlinks:] Wir beheben obiges Manko, indem wir jeder Page nur eine Gesamtstimme zugestehen. Daher erh�hen wir f�r eine Page $j$ den Stimmenanteil f�r eine andere Page um das Verh�ltnis $\frac{x_j}{n_j}$. Sei $L_k$ die Menge aller Backlinks von Page $k$:

\begin{equation*}\label{score}
x_k=\sum_{j\in L_k}\frac{x_j}{n_j}
\end{equation*}
Wir z�hlen Links auf sich selbst nicht; man kann sich also nicht selbst w�hlen.
\end{description}
%	  \end{column}
%	  \begin{column}{0.3\textwidth}            
		\tikzstyle{every entity} = [top color=white, bottom color=green!30, 
                            draw=green!50!black!100, minimum height={2.12cm}, minimum width={1.5cm} ,drop shadow]
                            
\tikzset{
>=triangle 45,
rot/.style={draw=black,thick
}
}
\tikzstyle{output3} = [coordinate] 
\tikzstyle{output4} = [coordinate] 
\tikzstyle{output1} = [coordinate]                            

\begin{figure}
\begin{center}
\scalebox{0.8}{
{\begin{tikzpicture}[node distance=1.5cm, every edge/.style={link}]

  \node[entity] (p1) {$P_1$};

  \node[entity] (p3) [right =of p1] {$P_3$} edge [<-] (p1);
  
  \node[entity] (p4) [below =of p3] {$P_4$} edge [->] (p3);
  
  \node[entity] (p2) [left =of p4] {$P_2$} edge [->] (p4);
                            
  \node [output3, left =0cm of p3.160] (output3) {};
  \draw[rot][<-] (p1.20) -- (output3);
  
  \node [output4, left =0cm of p4.115] (output4) {};
  \draw[rot][<-] (p1.320) -- (output4);
  
  \node [output1, left =0cm of p1.295] (output1) {};
  \draw[rot][<-] (p4.140) -- (output1);
  
   \draw[rot][->] (p1) -- (p2); 
   \draw[rot][->] (p2.55) -- (p3.235);                         

 \end{tikzpicture}}
}
\end{center}
\caption{Beispiel eines Webs mit 4 Pages. Die Pfeile stellen Links dar.}\label{kleinesbsp}
\end{figure}

Wir gehen dem zweiten Ansatz, mit einer kleinen Modifikation, nach.

%	  \end{column}
%	  \end{columns}
            \end{block}
\vskip1.5ex           
                        \begin{block}{Model \glqq gewichtete Backlinks\grqq}
F�r unser Beispielweb haben wir also die Scores
$x_1 = \frac{x_3}{1}+\frac{x_4}{2}$, $x_2 = \frac{x_3}{3}$, $x_3 = \frac{x_1}{3}+\frac{x_2}{2}+\frac{x_4}{2}$ und $x_4 = \frac{x_1}{3}+\frac{x_2}{2}$,
oder in Matrixform $\B{Ax}=\B{x}$ mit $\B{x}=[x_1\;x_2\;x_3\;x_4]^{T}$ und
$$\B{A}=\begin{pmatrix}
0 & 0 & 1& \frac{1}{2}\\[0.5ex]
\frac{1}{3} & 0 & 0 & 0\\[0.5ex]
\frac{1}{3} & \frac{1}{2} & 0 & \frac{1}{2}\\[0.5ex]
\frac{1}{3} & \frac{1}{2} & 0 & 0
\end{pmatrix}.$$

Das PageRanking-Problem ist also im Grunde nichts anderes als das Finden eines Eigenvektors f�r eine quadratische Matrix; und zwar einen mit Eigenwert $1$.

            \end{block}  
            
\vskip1.5ex            
	}
        \end{minipage}
      \end{beamercolorbox}
    \end{column}
    % ---------------------------------------------------------%
    % end the column

    % ---------------------------------------------------------%
    % Set up a column 
    \begin{column}{.49\textwidth}
      \begin{beamercolorbox}[center,wd=\textwidth]{postercolumn}
        \begin{minipage}[T]{.95\textwidth} % tweaks the width, makes a new \textwidth
          \parbox[t][\columnheight]{\textwidth}{ % must be some better way to set the the height, width and textwidth simultaneously
            % Since all columns are the same length, it is all nice and tidy.  You have to get the height empirically
            % ---------------------------------------------------------%
            % fill each column with content
\vskip1.5ex                        
      
            \begin{block}{Plausibilit�t des gew�hlten Models}
Unsere Beispielmatrix $\B{A}$ hat einen Eigenwert $1$ mit Eigenvektor $[12\;4\;9\;6]^{T}$. Wir interpretieren die Scores als Wahrscheinlichkeiten, dass ein Surfer eine Page besucht. Gerundet haben wir
$$\B{x}\approx[0.39\;0.13\;0.29\;0.19]^{T}.$$

Alles sch�n und gut; aber hat den f�r ein beliebiges Web die zugeh�rige Link-Matrix einen Eigenwert $1$?            \end{block}
\vskip2.5ex           
	\begin{block}{Probleme}
	\begin{itemize}
	\item Nicht-vernetzte Seiten (lassen wir ausser acht)
	\item Vergleichbarkeit von in sich geschlossenen \glqq Subwebs\grqq
	\end{itemize}
\tikzstyle{every entity} = [top color=white, bottom color=green!30, 
                            draw=green!50!black!100, minimum height={2.12cm}, minimum width={1.5cm} ,drop shadow]
                            
\tikzset{
>=triangle 45,
rot/.style={draw=black,thick
}
}
\tikzstyle{helpnode} = [coordinate] 
\tikzstyle{output4} = [coordinate] 
\tikzstyle{output1} = [coordinate]                            

\begin{figure}
\begin{center}
\scalebox{0.8}{
{\begin{tikzpicture}[node distance=1.5cm, every edge/.style={link}]

  \node[entity] (p1) {$P_1$};

  \node[entity] (p2) [below =of p1] {$P_2$} edge [->] (p1);
  
  \node[entity] (p3) [right = 1.3cm of p1] {$P_3$};
  
  \node[entity] (p4) [below =of p3] {$P_4$} edge [->] (p3);
                            
  \node [helpnode, right =1.8cm of p3] (helpnode) {};
 
  \node[entity] (p5) [below =0.8cm of helpnode] {$P_5$} edge [->] (p3.0);
  
  \draw[rot][->] (p5.225) -- (p4.0); 
  \draw[rot][->] (p1.250) -- (p2.110);
  \draw[rot][->] (p3.250) -- (p4.110);                         

 \end{tikzpicture}}
}
\end{center}
\caption{Ein Web mit 5 Pages, bestehend aus 2 Subwebs.}\label{bspnonunique}
\end{figure}

Hier haben wir die Link-Matrix
$$\B{A}=
\begin{pmatrix}
0 & 1 & 0 & 0 & 0\\
1 & 0 & 0 & 0 & 0\\
0 & 0 & 0 & 1 & \frac{1}{2}\\
0 & 0 & 1 & 0 & \frac{1}{2}\\
0 & 0 & 0 & 0 & 0
\end{pmatrix}$$
wobei der Eigenraum $V_1(\B{A})$ die Dimension $2$ hat; �bel! Die Frage, welchen dieser Eigenvektoren --- oder Linearkombination --- wir f�r das Ranking w�hlen, bleibt offen.
            \end{block}
\vskip2.5ex             
            \begin{block}{Eindeutiges Ranking und Vergleich von Subwebs}
F�r ein $n$-seitiges Web ohne lose Pages k�nnen wir ein eindeutiges Ranking schaffen. Sei $\B{S}$ eine $n\times n$ Matrix voller Komponenten $\frac{1}{n}$. F�r $m\in(0,1]$ ersetzen wir $\B{A}$ durch die gewichtete Matrix
\begin{equation*}\label{equM}
\B{M}=(1-m)\B{A}+m\B{S}.
\end{equation*}
Nun ist $V_1(\B{M})$ immer eindimensional und liefert folglich eindeutige Rankings. Google verwendet $m=0.15$.

F�r das 5-Web erhalten wir $\B{x}=[0.2\;0.2\;0.285\;0.285\;0.03]^{T}.$

Aktuell besteht das WWW aus mindestens 8 Milliarden Pages. Wie berechnen wir in vern�nftiger Zeit ein Ranking?  Nat�rlich iterativ.
            \end{block}
\vskip2.5ex             
            \begin{block}{Power Method}
Wir k�nnen beweisen, dass die Matrix $\B{M}$ f�r ein Web ohne lose Pages hat einen eindeutig bestimmten Ranking-Vektor. Dieser kann als Grenzwert der Iteration $\B{x}_k=(1-m)\B{Ax}_{k-1}+m\B{s}$ mit Initialvektor $\B{x}_0$ berechnet werden.

Am Beispiel von oben f�r $k$ Iterationen mit Initialvektor $\B{x}_0=[0.24\;0.31\;0.08\;0.18\;0.19]^{T}$ tabelliert Mathematica:

\begin{center}
\scalebox{0.8}{
\begin{tabular}{c|c|c|c}
$k$	&	$x_k$	&	$\|\B{M}^k\B{x}_0-\B{q}\|_1$	&	$\frac{\|\B{M}^k\B{x}_0-\B{q}\|_1}{\|\B{M}^{k-1}\B{x}_0-\B{q}\|_1}$	\\ \hline
\spaltenheight$0$	&	$\{0.240,0.310,0.080,0.180,0.190\}	&	$0.62$	&		\\
$1$	&	$\{0.294,0.234,0.264,0.179,0.030\}	&	$0.255$	&	$0.411$	\\
$5$	&	$\{0.249,0.218,0.274,0.230,0.030\}	&	$0.133$	&	$0.85$	\\
$10$	&	$\{0.208,0.222,0.260,0.280,0.030\}	&	$0.0591$	&	$0.85$\\
$50$	&	$\{0.200,0.200,0.285,0.285,0.030\}	&	$8.87\cdot10^{-5}$	&	$0.85$
\end{tabular}
}
\end{center}
Imba! Mathematics makes the Web go 'round!
            \end{block}
\vskip1.5ex
          }
          % ---------------------------------------------------------%
          % end the column
        \end{minipage}
      \end{beamercolorbox}
    \end{column}
    % ---------------------------------------------------------%
    % end the column
  \end{columns}
  %\tiny\hfill\textcolor{ta2gray}{Created with \LaTeX \texttt{beamerposter}  \url{http://www-i6.informatik.rwth-aachen.de/~dreuw/latexbeamerposter.php}}
  \tiny\hfill{Created with \LaTeX \texttt{beamerposter}  \url{www.koeniz-lerbermatt.ch} \hskip1em}
\end{frame}
\end{document}


%%%%%%%%%%%%%%%%%%%%%%%%%%%%%%%%%%%%%%%%%%%%%%%%%%%%%%%%%%%%%%%%%%%%%%%%%%%%%%%%%%%%%%%%%%%%%%%%%%%%
%%% Local Variables: 
%%% mode: latex
%%% TeX-PDF-mode: t
%%% End: